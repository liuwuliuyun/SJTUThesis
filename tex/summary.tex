%# -*- coding: utf-8-unix -*-
%%==================================================
%% conclusion.tex for SJTUThesis
%% Encoding: UTF-8
%%==================================================

\begin{summary}

本文针对无感知环境下的大规模人脸数据的模糊搜索,从五个方面开展研究:

第一,建立无感知人脸检测的数据集CFDDB。并在CFDDB上定义召回率$recall$、误判率$errorrate$与检测参数$\alpha$等测试基准。之后进行人脸数据属性分析,探讨数据集的改进方法,制定数据集的扩充流程。

第二,根据CFDDB测试集的基准定义,完成四种不同的人脸检测算法的封装测试与参数调优。根据测试结果,选择SSH检测器\cite{najibi2017ssh}作为系统人脸检测模块的核心。

第三,利用CFDDB中的数据,对InsightFace\cite{deng2018arcface}进行测试,并将其封装为系统的人脸识别模块。

第四,针对小规模数据匹配,提出线性搜索和多类支持向量机两种方案。针对大规模数据匹配,研究基于局部敏感哈希与基于有向图的两种近似搜索算法,并提出优化思路。

第五,高效、稳定、可扩展的系统架构设计。

综合以上五方面的研究成果,最终在GPU服务器上实现了无感知人脸识别系统。

\end{summary}
