\chapter{人脸检测算法}
\label{chap:facedetection}

人脸检测算法自上世纪以来已经被无数研究者探究过,检测的准确率也是逐年提高。在本系统中,我们需要的是一个能够优秀的处理大角度人脸、模糊人脸甚至部分被遮挡人脸的检测算法。在保证非常高的准确率的同时,该算法需要有非常快的执行速度,以适应同时处理大量图片的需求。

CFDDB测试集中全部的人脸数量为$face_{all}$,检测正确的人脸数量为$face_{right}$,定义检测率$r$为:

\begin{displaymath}
\label{eq:rdef}
	r = \frac{face_{right}}{face_{all}} 
\end{displaymath}

以CFDDB中的图片为标准,我们要求人脸检测算法需要在 定义的$r\geq 80\%$的情况下,在一张Nvidia 1080 Ti显卡的支持下处理一张宽1600像素,高1200像素的图片的平均速度不超过\SI{250}{ms}。

为了简化测试过程,我们要求在正式测试前,人脸检测算法需要在CFDDB中随机抽取的五张图片中$r\geq 75\%$。


\section{基于HAAR特征的级联分类器}

\subsection{概述}
在2001年,Paul Viola和Michael Jones在论文[9]中提出了使用HAAR特征来识别人脸的方法。HAAR特征是



\subsection{在CFDDB上的测试结果}


\section{HOG}

\subsection{概述}

\subsection{在CFDDB上的测试结果}


\section{SSH}

\subsection{概述}

\subsection{在CFDDB上的测试结果}

